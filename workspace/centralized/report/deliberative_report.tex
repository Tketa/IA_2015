
\documentclass[fontsize=12pt]{scrartcl} % A4 paper and 12pt font size

\usepackage[T1]{fontenc} % Use 8-bit encoding that has 256 glyphs
%\usepackage{fourier} % Use the Adobe Utopia font for the document - comment this line to return to the LaTeX default
\usepackage{spverbatim}
\usepackage[bottom=6em]{geometry}
\usepackage[english]{babel} % English language/hyphenation
\usepackage[babel=true]{csquotes} 
\usepackage{amsmath,amsfonts,amsthm} % Math packages
\usepackage{lipsum} % Used for inserting dummy 'Lorem ipsum' text into the template
\usepackage{graphicx}
\usepackage{titlepic}
\usepackage{bbm}
\usepackage{color}
\usepackage{sectsty} % Allows customizing section commands
\usepackage{listings} %Allows Java code
\setlength{\headheight}{0pt} % Customize the height of the header


%\numberwithin{equation}{section} % Number equations within sections (i.e. 1.1, 1.2, 2.1, 2.2 instead of 1, 2, 3, 4)
%\numberwithin{figure}{section} % Number figures within sections (i.e. 1.1, 1.2, 2.1, 2.2 instead of 1, 2, 3, 4)
%\numberwithin{table}{section} % Number tables within sections (i.e. 1.1, 1.2, 2.1, 2.2 instead of 1, 2, 3, 4)


%----------------------------------------------------------------------------------------
%	TITLE SECTION
%----------------------------------------------------------------------------------------

\newcommand{\horrule}[1]{\rule{\linewidth}{#1}} % Create horizontal rule command with 1 argument of height
\title{	
 %  \includegraphics[width=4cm]{lia-logo.jpg} % also works with logo.pdf
\normalfont \normalsize 
\textsc{Intelligent Agents, EPFL} \\ [20pt] % Your university, school and/or department name(s)
\horrule{0.5pt} \\[0.4cm] % Thin top horizontal rule
\huge Centralized agents \\ % The assignment title
\horrule{2pt} \\[0.5cm] % Thick bottom horizontal rule
}
\author{Jeremy Gotteland \& Quentin Praz} % Your name
\date{\normalsize\today} % Today's date or a custom date
\begin{document}
\maketitle % Print the title

%----------------------------------------------------------------------------------------
%	PROBLEMS SECTION
%----------------------------------------------------------------------------------------

\section*{Encoding the problem}
We create the class $ExtendedTask$ which is the extension of a basic task provided by \textit{Logist}. This new class is used to differentiate the pickup action and the delivery action. It is characterize by a task and by an action (pickup or delivery). We used these $ExtendedTask$s to be able to carry multiple tasks in the same vehicle.\\
We also implement the class $Solution$ which represents a possible solution. A solution is characterized by a map between a vehicle and a list of $ExtendedTask$s that we called $tasksList$. We have therefore one $tasksList$ per vehicle.\\
The notion of time is define by the indices of the task in the $tasksList$.

\section*{Constraints}
To have a valid solution, we need to satisfy constraints:
\begin{enumerate}
\item A specific $ExtendedTask$ can be contained in only one $tasksList$.
\item Let take the two $ExtendedTask$s coming from the same task (pickup and delivery part), the index of the pickup part has to be strictly smaller that the index of the delivery part.
\item Let take again the two $ExtendedTask$s coming from the same task, the two $ExtendedTask$ have to be in the same $tasksList$.
\item All task must be deliver. The sum of the $ExtendedTask$s from all $tasksList$s had to be equal to the number of initial tasks multiplied by 2.
\item The weights of the carried tasks by a vehicle cannot exceed the capacity of this vehicle. 
\end{enumerate}


\end{document}          
