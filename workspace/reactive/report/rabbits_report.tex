
\documentclass[fontsize=11pt]{scrartcl} % A4 paper and 11pt font size

\usepackage[T1]{fontenc} % Use 8-bit encoding that has 256 glyphs
%\usepackage{fourier} % Use the Adobe Utopia font for the document - comment this line to return to the LaTeX default
\usepackage[bottom=6em]{geometry}
\usepackage[english]{babel} % English language/hyphenation
\usepackage[babel=true]{csquotes} 
\usepackage{amsmath,amsfonts,amsthm} % Math packages
\usepackage{lipsum} % Used for inserting dummy 'Lorem ipsum' text into the template
\usepackage{graphicx}
\usepackage{titlepic}
\usepackage{bbm}
\usepackage{color}
\usepackage{sectsty} % Allows customizing section commands
\usepackage{listings} %Allows Java code
\setlength{\headheight}{0pt} % Customize the height of the header

%\numberwithin{equation}{section} % Number equations within sections (i.e. 1.1, 1.2, 2.1, 2.2 instead of 1, 2, 3, 4)
%\numberwithin{figure}{section} % Number figures within sections (i.e. 1.1, 1.2, 2.1, 2.2 instead of 1, 2, 3, 4)
%\numberwithin{table}{section} % Number tables within sections (i.e. 1.1, 1.2, 2.1, 2.2 instead of 1, 2, 3, 4)


%----------------------------------------------------------------------------------------
%	TITLE SECTION
%----------------------------------------------------------------------------------------

\newcommand{\horrule}[1]{\rule{\linewidth}{#1}} % Create horizontal rule command with 1 argument of height
\title{	
 %  \includegraphics[width=4cm]{lia-logo.jpg} % also works with logo.pdf
\normalfont \normalsize 
\textsc{Intelligent Agents, EPFL} \\ [20pt] % Your university, school and/or department name(s)
\horrule{0.5pt} \\[0.4cm] % Thin top horizontal rule
\huge Reactive agents \\ % The assignment title
\horrule{2pt} \\[0.5cm] % Thick bottom horizontal rule
}
\author{Jeremy Gotteland \& Quentin Praz} % Your name
\date{\normalsize\today} % Today's date or a custom date
\begin{document}
\maketitle % Print the title

%----------------------------------------------------------------------------------------
%	PROBLEMS SECTION
%----------------------------------------------------------------------------------------

\section*{World representation}
\subsection*{State representation}
Recall that the agent here refers to a single vehicle.
We chose to represent the state of an agent as the following:
\begin{itemize}
\item Current city: The city in which the agent is currently located.
\item Destination city: At any time the agent is a state where it could pick up a task to bring to a city. This variable is this potential city. If there is no task to be pick up, this variable is set to $NULL$.
\item Possible moves: A list of cities which the agent can travel to from current state i.e. its neighbor cities and its potential destination city.
\end{itemize}

\subsection*{Actions}
In our state representation, an action is just a move to another city.

\subsection*{Reward}
The reward for moving from a state to another city $c$ is defined as:
$$
R(s,c) = \mathbbm{1}_{ \{ \textit{c = s.destinationCity} \} } \cdot AR(s.currentCity, c) - Km(s.currentCity, c) \cdot CostPerKm
$$

\subsection*{Probability of transition}
The probability of transition to the state $s'$ from the state $s$ by moving to city $c$ is defined as:
$$
T(s,c,s') = 
\left\{
\begin{array}{l}
 P(c, s'.destinationCity) \text{ if $destinationCity$ is not $NULL$}\\
 1-\sum_{c' \in Cities} P(c,c') \text{ otherwise}
\end{array}
\right.
$$
\section*{Algorithm}
Here we describe the algorithm we used
\section*{Implementation details}


\section*{Observations}

\end{document}          
